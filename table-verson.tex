%%
%% Example of plain layout for neoresponse.sty
%% 
%% Copyright (C) 2021 Nathan Wu. 
%% Written by Nathan Wu.
%% Distributed under the terms of the CC-BY-SA v3.0 Licence
%%
%% 
\documentclass[12pt]{article}
\usepackage[utf8]{inputenc}
\usepackage{color}

\usepackage[cm]{fullpage}
\usepackage{fancyhdr}
\fancypagestyle{mypage}{
	\fancyfoot[C]{\thepage \\ \textcolor[RGB]{96,96,96}{PLEASE KEEP CONFIDENTIAL} }
	}
\pagestyle{mypage}

\renewcommand{\headrulewidth}{0pt}

\usepackage{longtable}

\usepackage[table]{neoresponse}

\usepackage[parfill]{parskip}
\usepackage{lipsum}

\begin{document}

\begin{center}
	\section*{\Huge Response to reviewers’ comments on manuscript}	
\end{center}

\hrule


% --- basic letter

%Dear Journal Editor
%
%Thank you for considering our paper ``A Journal Article'' for publication in the ``Journal of Something''. We are very grateful for the time the reviewers have taken to provide their extensive and constructive feedback.
%
%We address their points individually below and indicate the changes we have made in response.
%
%Yours sincerely

\hspace{3cm}

%A Researcher


% First reviewer
\reviewersection

\begin{longtable}{p{10.5cm}p{7cm}}
\begin{comment}
  \lipsum[2]
  \label{C1:Point}
\end{comment}
&
\begin{response}
Thank you for drawing our attention to the omission. We have added both references as suggested.
\end{response}  
\\
\begin{comment}
\lipsum[11]
\end{comment}
&
\begin{response}
Thank you for the suggestion. The second reviewer makes a related point and we address your comments in detail in our response to C\ref{C1:Point}
\end{response}
\\
\end{longtable}

\clearpage


% Start a new reviewer section
\reviewersection

\begin{longtable}{p{10.5cm}p{7cm}}
\begin{comment}
  \lipsum[8][1-4]
\end{comment}
&
\begin{response}
\lipsum[9][1-3] See also our response to C\ref{C1:Point}
\end{response}
\\
\begin{comment}
  \label{C2:Point}
  \lipsum[5]
\end{comment}
&
\begin{response}
  \lipsum[4]
\end{response}
\\
\end{longtable}


\end{document}